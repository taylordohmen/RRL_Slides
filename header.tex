% \usepackage{atveryend}
\usepackage{varwidth}
\usepackage{bookmark}
\usepackage{microtype}
\usepackage[numbers]{natbib}

\usepackage{graphicx}
\usepackage{caption}
\captionsetup[figure]{labelformat = empty}
\usepackage{subcaption}
\captionsetup[subfigure]{labelformat = empty}

\usepackage{tikz}
\usetikzlibrary{
    % spy,
    arrows,
    arrows.meta,
    automata,
    calc,
    % decorations.markings,
    fit,
    graphs,
    hobby,
    intersections,
    % patterns,
    positioning,
    quotes,
    shapes.geometric,
    shapes.misc
}

% AMSMATH IS PICKY ABOUT BEING LOADED BEFORE SOME OTHER PACKAGES
\usepackage{amsmath}

% THEOREMS, DEFINITIONS, EXAMPLES, ETC.
\usepackage{amsthm}
\usepackage[theorems, skins]{tcolorbox}

\usepackage{mathtools}
\usepackage{fourier-otf}
\usepackage{stmaryrd}

% LISTS
\usepackage[inline]{enumitem}

\setlist[itemize]{
    leftmargin = 15pt,
    itemsep = 1ex,
    % parsep = 0.25ex,
    topsep = 1ex,
    label = \raisebox{0.33ex}{\scalebox{0.75}{$\blacktriangleright$}}
}

\setlist[enumerate]{
    leftmargin = 15pt,
    itemsep = 1ex,
    % parsep = 0.25ex,
    topsep = 1ex,
    label = (\arabic*)
}


% TABLES
\usepackage{booktabs}
\usepackage{makecell}


% BEAMER JUNK
\useoutertheme[
    numbering = fraction,
    progressbar = frametitle
]{metropolis}

\useinnertheme[
    sectionpage = progressbar,
    subsectionpage = progressbar
]{metropolis}

\usefonttheme[titleformat title = regular]{metropolis}
\usefonttheme[onlymath]{serif}
\setsansfont[BoldFont={Fira Sans SemiBold}]{Fira Sans Book}
\setbeamerfont{page number in head/foot}{size = \tiny}

\usecolortheme{spruce}
% \usecolortheme{spruce}
% \usefonttheme{serif}

\setbeamersize{
    text margin left = 15pt,
    text margin right = 15pt
}

\defbeamertemplate{section page}{myprogressbar}{
    \centering
    \begin{minipage}{\textwidth}
        \raggedright
        \usebeamercolor[fg]{section title}
        \usebeamerfont{section title}
        \insertsectionhead\\[-1ex]
        \usebeamertemplate*{progress bar in section page}
        \mbox{}\\
        \par
    \end{minipage}
    \par
    \vspace{\baselineskip}
}

\defbeamertemplate{subsection page}{myprogressbar}{
    \centering
    \begin{minipage}{\textwidth}
        \raggedright
        \usebeamercolor[fg]{section title}
        \usebeamerfont{section title}
        \insertsectionhead\\[-1ex]
        \usebeamertemplate*{progress bar in section page}
        \mbox{}\\
        \par
        \ifx\insertsubsectionhead\@empty\else%
        \usebeamercolor[fg]{subsection title}%
        \usebeamerfont{subtitle}%
        \tableofcontents[sectionstyle = hide/hide, subsectionstyle = show/shaded/hide]
        \fi
    \end{minipage}
    \par
    \vspace{\baselineskip}
}

\setbeamertemplate{section page}[myprogressbar]
\setbeamertemplate{subsection page}[myprogressbar]
% \setbeamertemplate{background}[grid]
% \setbeameroption{show notes}

% FONTS
\newcommand{\tu}[1]{\textup{#1}}
\newcommand{\tb}[1]{\textbf{#1}}
\newcommand{\tn}[1]{\tu{#1}}
\newcommand{\ts}[1]{\textsf{#1}}
\newcommand{\tsc}[1]{\textsc{#1}}
\newcommand{\bb}[1]{\ensuremath{\mathbb{#1}}}
\newcommand{\ms}[1]{\ensuremath{\mathsf{#1}}}
\newcommand{\mc}[1]{\ensuremath{\mathcal{#1}}}
\newcommand{\mb}[1]{\ensuremath{\mathbf{#1}}}
\renewcommand{\vec}[1]{\ensuremath{\bm{#1}}}
\newcommand{\mf}[1]{\ensuremath{\mathfrak{#1}}}
\newcommand{\mt}[1]{\ensuremath{\mathtt{#1}}}
\newcommand{\msc}[1]{\ensuremath{\mathscr{#1}}}


% PAIRED DELIMITERS
\DeclarePairedDelimiter\set{\lbrace}{\rbrace}
\DeclarePairedDelimiter\sem{\llbracket}{\rrbracket}
\DeclarePairedDelimiter\paren{(}{)}
\DeclarePairedDelimiter\tup{\langle}{\rangle}
\DeclarePairedDelimiter\seq{(}{)}
\DeclarePairedDelimiter\card{\lvert}{\rvert}
\DeclarePairedDelimiter\brak{[}{]}
% \DeclarePairedDelimiter\numsem{\llangle}{\rrangle}
\DeclarePairedDelimiter\floor{\lfloor}{\rfloor}

% PAST-DISCOUNTING STUFF
% \newcommand{\last}[1]{\ms{last}\paren*{#1}}
\newcommand{\cond}[1]{\tu{Cnd}\paren*{#1}}
\newcommand{\icond}[2]{\tu{Cnd}^\top_{#1,\flat}\paren*{#2}}
\newcommand{\scond}[2]{\tu{Cnd}^\top_{#1,\sharp}\paren*{#2}}
\newcommand{\val}[3]{\tu{Val}_{#2}\paren*{#1, #3}}
\newcommand{\pval}[4]{\tu{Val}^{#4}_{#2}\paren*{#1, #3}}
\newcommand{\Path}[1]{\tu{Path}\paren*{#1}}
\newcommand{\Traj}[1]{\tu{Traj}\paren*{#1}}
\newcommand{\xPath}[2]{\tu{Path}_{#2}\paren*{#1}}
\newcommand{\xTraj}[2]{\tu{Traj}_{#2}\paren*{#1}}
\newcommand{\sPath}{\tu{Path}}
\newcommand{\sTraj}{\tu{Traj}}
\newcommand{\sbsq}[3]{#1{\scriptstyle\brak*{#2, #3}}}
\DeclareMathOperator*{\limopt}{\tu{lim}~\tu{opt}}
\newcommand{\shortflat}{\scalebox{1}[.66]{$\flat$}}
\newcommand{\uPD}{\ms{P}^\sharp_\gamma}
\newcommand{\lPD}{\ms{P}^{\shortflat}_\gamma}
\newcommand{\PD}{\ms{P}^\natural_\gamma}
\newcommand{\transpose}[1]{{#1}^{\intercal}}


% STANDARDIZED ABBREVIATIONS
\newcommand{\eg}{\emph{e.g.}}
\newcommand{\cf}{\emph{cf.}}
\newcommand{\ie}{\emph{i.e.}}
\newcommand{\etc}{\emph{etc.}}

% CONVENIENCE
\newcommand{\w}{\ensuremath{\omega}}
\newcommand{\nth}[1]{{#1}^{\tu{th}}}
\newcommand{\bigast}{\mathop{\scalebox{1.75}{\raisebox{-0.2ex}{$\ast$}}}}
\newcommand{\largediamond}{\mathop{\scalebox{1.75}{\raisebox{-0.2ex}{$\diamond$}}}}
\newcommand{\Emph}[1]{\emph{\textrm{#1}}}

% COMPLEXITY CLASSES
\newcommand{\dspace}[1]{\textrm{DSPACE}\ensuremath{\paren*{#1}}}

\tikzset{
    every picture/.style = {
        > = Stealth,
        thick,
        initial text =
    },
    loop/.append style = {
        looseness = 5,
        min distance = 0.5cm
    },
    loop above/.append style = {out = 110, in = 70 },
    loop below/.append style = {out = 290, in = 250},
    loop left/.append  style = {out = 200, in = 160},
    loop right/.append style = {out = 20,  in = 340},
    every state/.style = {
        inner sep = 2pt,
        outer sep = 0pt,
        minimum size = 15pt
    },
    every node/.style = {
        inner sep = 1pt,
        outer sep = 2pt,
        minimum size = 0pt,
        fill = white,
        align = center
    },
    graphs/edge quotes = {
        inner sep = 1pt,
        outer sep = 2pt,
        fill = white
    },
    dots/.style args = {#1per #2}{%
        line cap=round,
        dash pattern=on 0 off #2/#1
    }
}

% COLORS
\newcommand{\red}[1]{{\color{red}#1}}
\newcommand{\blue}[1]{{\color{blue}#1}}
\newcommand{\teal}[1]{{\color{teal}#1}}
\newcommand{\violet}[1]{{\color{Violet}#1}}
\newcommand{\purple}[1]{{\color{purple}#1}}
\newcommand{\grey}[1]{{\color{gray}#1}}
\newcommand{\cyan}[1]{{\color{cyan}#1}}
\newcommand{\green}[1]{{\color{Green}#1}}
\newcommand{\magenta}[1]{{\color{magenta}#1}}
\newcommand{\rblue}[1]{{\color{RoyalBlue}#1}}

% THEOREM-LIKE ENVIRONMENTS
\DeclareTcbTheorem[no counter]{theorem}{Theorem}{
    enhanced,
    frame empty,
    % interior empty,
    colback = Cerulean!5!white,
    colframe = Cerulean!50!white,
    coltitle = black,
    fonttitle = \footnotesize\bfseries,
    colbacktitle = Cerulean!25!white,
    borderline = {0.5mm}{0mm}{Cerulean!50},
    attach boxed title to top left = {
        xshift = 5pt,
        yshift = -5pt
    },
    boxed title style = {boxrule = 1pt},
    varwidth boxed title,
    label separator = {:},
    separator sign colon,
    fontupper = \footnotesize
}{thm}

% \DeclareTcbTheorem[no counter]{definition}{Definition}{
%     enhanced,
%     frame empty,
%     interior empty,
%     colframe = MidnightBlue!50!white,
%     coltitle = black,
%     fonttitle = \bfseries,
%     colbacktitle = MidnightBlue!15!white,
%     borderline = {0.5mm}{0mm}{MidnightBlue!50},
%     attach boxed title to top left = {
%         xshift = 5pt,
%         yshift = -5pt
%     },
%     boxed title style = {boxrule = 1pt},
%     varwidth boxed title,
%     label separator = {:},
%     separator sign colon,
%     fontupper = \footnotesize
% }{def}

\DeclareTcbTheorem[no counter]{corollary}{Corollary}{
    enhanced,
    frame empty,
    % interior empty,
    colback = Apricot!5!white,
    colframe = Apricot!50!white,
    coltitle = black,
    fonttitle = \footnotesize\bfseries,
    colbacktitle = Apricot!25!white,
    borderline = {0.5mm}{0mm}{Apricot!75},
    attach boxed title to top left = {
        xshift = 5pt,
        yshift = -5pt
    },
    boxed title style = {boxrule = 1pt},
    varwidth boxed title,
    label separator = {:},
    separator sign colon,
    fontupper = \footnotesize
}{cor}


% \tikzexternalize[prefix = figures/]
% % INJECTING CODE BEFORE/AFTER SPECIFIC ENVIRONMENTS
% \BeforeBeginEnvironment{definition}{\tikzexternaldisable}
% \AfterEndEnvironment{definition}{\tikzexternalenable}
% \BeforeBeginEnvironment{theorem}{\tikzexternaldisable}
% \AfterEndEnvironment{theorem}{\tikzexternalenable}
% \BeforeBeginEnvironment{corollary}{\tikzexternaldisable}
% \AfterEndEnvironment{corollary}{\tikzexternalenable}


\renewcommand{\thefootnote}{\raisebox{1ex}{\tiny\fnsymbol{footnote}}}

\usepackage{bm}